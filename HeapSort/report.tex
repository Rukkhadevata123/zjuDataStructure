\documentclass[UTF8]{ctexart}
\usepackage{geometry, CJKutf8}
\geometry{margin=1.5cm, vmargin={0pt,1cm}}
\setlength{\topmargin}{-1.64723cm} % 调整 topmargin
\setlength{\headheight}{12.64723pt} % 设置 headheight
\setlength{\paperheight}{29.7cm}
\setlength{\textheight}{25.3cm}

% useful packages.
\usepackage{amsfonts}
\usepackage{listingsutf8}
\usepackage{amsmath}
\usepackage{amssymb}
\usepackage{amsthm}
\usepackage{enumerate}
\usepackage{graphicx}
\usepackage{multicol}
\usepackage{fancyhdr}
\usepackage{layout}
\usepackage{listings}
\usepackage{float, caption}
\usepackage{fontspec}
\setmonofont{Noto Sans Mono}

\lstset{
    basicstyle=\ttfamily, basewidth=0.5em, inputencoding=utf8, showstringspaces=false, showspaces=false
}

% some common command
\newcommand{\dif}{\mathrm{d}}
\newcommand{\avg}[1]{\left\langle #1 \right\rangle}
\newcommand{\difFrac}[2]{\frac{\dif #1}{\dif #2}}
\newcommand{\pdfFrac}[2]{\frac{\partial #1}{\partial #2}}
\newcommand{\OFL}{\mathrm{OFL}}
\newcommand{\UFL}{\mathrm{UFL}}
\newcommand{\fl}{\mathrm{fl}}
\newcommand{\op}{\odot}
\newcommand{\Eabs}{E_{\mathrm{abs}}}
\newcommand{\Erel}{E_{\mathrm{rel}}}

\begin{document}

\pagestyle{fancy}
\fancyhead{}
\lhead{黎学圣, 3230102179}
\chead{数据结构与算法第七次作业}
\rhead{Nov.17th, 2024}

\section{堆排序核心代码}

我们分三步来实现堆排序算法。

\subsection{\texttt{sift\_down}函数}

\begin{lstlisting}[language=C++]
    template <typename T> 
    void sift_down(std::vector<T> &heap, int start, int end) {
        int parent = start;         // 父节点下标
        int child = 2 * parent + 1; // 左孩子下标
    
        while (child <= end) { // 防止越界
            if (child + 1 <= end && heap[child] < heap[child + 1]) {
                child += 1; // 选择较大的孩子
            }
    
            if (heap[parent] < heap[child]) {
                std::swap(heap[parent], heap[child]); // 调整父子节点
                parent = child;
                child = 2 * parent + 1;
            } else {
                break;
            }
        }
    }
\end{lstlisting}

我们先初始化父子节点,要交换子节点为父节点下标的两倍加一或两倍加二,
这由两个子节点的比较得出。(前提是没有越界)
然后我们不断比较父节点和子节点的大小,
如果父节点小于子节点,我们就交换两者的值,否则堆性质满足,退出循环。

\subsection{\texttt{heapify}函数}

\begin{lstlisting}[language=C++]
    template <typename T> 
    void heapify(std::vector<T> &heap) {
        int n = heap.size();
        for (int i = n / 2 - 1; i >= 0; --i) {
            sift_down(heap, i, n - 1);
        } // 从最后一个子节点的父节点开始调整
    }
\end{lstlisting}

这里\texttt{for}循环的起始位置是$\lfloor \frac{n}{2} \rfloor$,
意思是从最后一个子节点的父节点开始,逐渐向前遍历,完成堆的构建。

\subsection{\texttt{heap\_sort}函数}

\begin{lstlisting}[language=C++]
    template <typename T> 
    void heap_sort(std::vector<T> &arr) {
        Timer timer; // 开始计时
        heapify(arr); // 建堆
    
        for (int i = arr.size() - 1; i > 0; --i) {
            std::swap(arr[0], arr[i]); // 将堆顶元素与最后一个元素交换
            sift_down(arr, 0, i - 1);  // 调整堆
        }
        // 计时器在析构函数中自动停止并输出时间
    }
\end{lstlisting}

接下来会介绍辅助函数,包括计时器。开始计时后,进行堆的构建,
然后不断交换堆顶元素(即当前堆的最大元素)和最后一个元素,
然后调整堆,直到堆的大小为1,进行了原地的升序排序。
完成后计时器会自动停止并输出时间。

\section{辅助方法}

\subsection{计时器类}

\begin{lstlisting}[language=C++, breaklines=true]
    class Timer {
        public:
            Timer() : start_time_point(std::chrono::high_resolution_clock::now()) {}
        
            ~Timer() {
                stop();
            }
        
            void stop() {
                auto end_time_point = std::chrono::high_resolution_clock::now();
                auto start = std::chrono::time_point_cast<std::chrono::microseconds>(start_time_point).time_since_epoch().count();
                auto end = std::chrono::time_point_cast<std::chrono::microseconds>(end_time_point).time_since_epoch().count();
        
                auto duration = end - start;
                double seconds = duration * 0.000001;
        
                std::cout << std::fixed << std::setprecision(4) << "Duration: " << seconds << " seconds" << std::endl;
            }
        
        private:
            std::chrono::time_point<std::chrono::high_resolution_clock> start_time_point;
        };    
\end{lstlisting}

我们实现的计时器在析构函数中会自动停止并输出时间,
我们只需要在排序函数前创建一个计时器对象即可。函数执行完毕后,
计时器对象会自动析构,输出时间。

方法的关键是使用两个时间点构建\texttt{start}和\texttt{end}对象,相减后,转换为秒数输出,
结果为秒数保留四位小数。

\subsection{验证排序正确性}

\begin{lstlisting}[language=C++]
    template <typename T>
    bool check_sort(const std::vector<T> &arr) {
        for (size_t i = 1; i < arr.size(); ++i) {
            if (arr[i - 1] > arr[i]) {
                return false;
            }
        }
        return true;
    }
\end{lstlisting}

我们的程序是升序排序,这里只需检查相邻元素是否递增即可,如果有逆序对,
返回\texttt{false},否则返回\texttt{true}。

\subsection{与标准堆排序比较}

\begin{lstlisting}[language=C++]
    void stl_heap_sort(std::vector<T> &arr) {
        Timer timer; // 开始计时
        std::make_heap(arr.begin(), arr.end());
        std::sort_heap(arr.begin(), arr.end());
        // 计时器在析构函数中自动停止并输出时间
    }

    template <typename T>
    void compare(std::vector<T> &arr) {
        std::vector<T> arr1(arr);
        std::vector<T> arr2(arr);
    
        std::cout << "My Heap Sort: ";
        heap_sort(arr1);
    
        std::cout << "STL Heap Sort: ";
        stl_heap_sort(arr2);
    
        if (check_sort(arr1) && check_sort(arr2)) {
            std::cout << "Two sorting algorithms are correct." << std::endl;
        } else {
            std::cout << "Two sorting algorithms are wrong." << std::endl;
        }
        std::cout << std::endl;
    }
\end{lstlisting}

首先使用一个函数封装有计时器的STL堆排序,然后我们比较两种排序算法的结果,
最后我们检验了排序的正确性。

\subsection{测试辅助函数}

\begin{lstlisting}[language=C++]
    // 生成测试序列
    enum class Order {
        ASC, // 升序
        DESC, // 降序
        RANDOM // 随机
    };
    
    enum class Repeat {
        NO, // 无重复
        YES // 有重复
    };
\end{lstlisting}

这里是测试序列的类型,我们对整数和浮点数都进行了六次测试,使用枚举变量
排列组合。

\begin{lstlisting}[language=C++]
    template <typename T>
    void fill_data(std::vector<T>& arr, std::function<T()> generator, Repeat repeat) {
        if (repeat == Repeat::NO) {
            std::unordered_set<T> unique_elements;
            for (long long i = 0; i < arr.size(); ++i) {
                T value;
                do {
                    value = generator();
                } while (unique_elements.find(value) != unique_elements.end());
                unique_elements.insert(value);
                arr[i] = value;
            }
        } else {
            for (long long i = 0; i < arr.size(); ++i) {
                arr[i] = generator();
            }
        }
    }
\end{lstlisting}

这段代码我们用来填充数据,\texttt{generator}是一个返回随机数的函数,用于生成对应类型的序列。
稍后我们会看到,我们使用了
\texttt{std::uniform\_int\_distribution}
和\texttt{std::uniform\_real\_distribution}来生成随机数。
在\texttt{Repeat::NO}的情况下,我们使用\texttt{std::unordered\_set}来保证没有重复元素。
否则,我们直接生成随机数,插入到参数\texttt{arr}中。

下面是生成整数的方法:

\begin{lstlisting}[language=C++]
    std::vector<int> generate_data_int(long long size, Order order, Repeat repeat) {
        std::vector<int> arr(size);
        std::random_device rd;
        std::mt19937 gen(rd());
        std::uniform_int_distribution<int> dis(1, size);
    
        auto generator = [&]() { return dis(gen); };
        fill_data<int>(arr, generator, repeat);
    
        if (order == Order::ASC) {
            std::sort(arr.begin(), arr.end());
        } else if (order == Order::DESC) {
            std::sort(arr.begin(), arr.end(), std::greater<int>());
        }
    
        return arr;
    }
\end{lstlisting}

在C++中生成随机数,我们先随机数种子,比如这里的\texttt{std::random\_device},
然后用\texttt{std::mt19937}引擎生成随机数,最后用\texttt{std::uniform\_int\_distribution}
标准正态分布生成随机数。

填充数据方法的参数需要一个函数对象,这里我们使用了lambda表达式,
用刚刚创建的\texttt{dis}对象生成随机数,然后我们使用\texttt{fill\_data}函数填充数据。

接着检查\texttt{Order}枚举变量,如果是升序,我们使用标准可的排序算法进行排序
;如果是降序,我们增加一个比较函数\texttt{std::greater<int>()},然后排序。

返回的\texttt{arr}是一个生成的整数序列。

对于浮点数,只要把整数的正态分布对象改成浮点数的即可,
即\texttt{std::uniform\_real\_distribution}。

\section{进行测试}

下面是我们的测试代码: 

\begin{lstlisting}[language=C++]
    int main() {
        long long size = 1e7 + 5;
    
        // 测试不同排列组合
        std::vector<std::pair<Order, Repeat>> combinations = {
            {Order::ASC, Repeat::NO},
            {Order::ASC, Repeat::YES},
            {Order::DESC, Repeat::NO},
            {Order::DESC, Repeat::YES},
            {Order::RANDOM, Repeat::NO},
            {Order::RANDOM, Repeat::YES}
        };
    
        for (const auto& combination : combinations) {
            Order order = combination.first;
            Repeat repeat = combination.second;
    
            std::cout << "Testing with Order: " 
                      << (order == Order::ASC ? "ASC" : (order == Order::DESC ? "DESC" : "RANDOM"))
                      << ", Repeat: " 
                      << (repeat == Repeat::NO ? "NO" : "YES") 
                      << std::endl;
    
            std::cout << "Test integer:" << '\n';
            std::vector<int> data_int = generate_data_int(size, order, repeat);
            compare(data_int);
    
            std::cout << "Test double:" << '\n';
            std::vector<double> data_double = generate_data_double(size, order, repeat);
            compare(data_double);
        }
    
        return 0;
    }
\end{lstlisting}

作业要求至少1e6的数据,这里我们使用了1e7+5的数据量。

我们对六种排列组合进行测试,每种情况下,我们生成整数和浮点数序列,
每次测试都有对应结果。

我们使用\texttt{clang++ test.cpp -o test -std=c++20 -O2}
来进行编译,然后运行\texttt{./test}进行测试。

完成测试总共大概需要一分半到两分钟。

以下是某一次的测试输出:

\begin{verbatim}
    Testing with Order: ASC, Repeat: NO
    Test integer:
    My Heap Sort: Duration: 0.5548 seconds
    STL Heap Sort: Duration: 0.4965 seconds
    Two sorting algorithms are correct.
    
    Test double:
    My Heap Sort: Duration: 0.6206 seconds
    STL Heap Sort: Duration: 0.4936 seconds
    Two sorting algorithms are correct.
    
    Testing with Order: ASC, Repeat: YES
    Test integer:
    My Heap Sort: Duration: 0.5784 seconds
    STL Heap Sort: Duration: 0.4909 seconds
    Two sorting algorithms are correct.
    
    Test double:
    My Heap Sort: Duration: 0.6434 seconds
    STL Heap Sort: Duration: 0.4913 seconds
    Two sorting algorithms are correct.
    
    Testing with Order: DESC, Repeat: NO
    Test integer:
    My Heap Sort: Duration: 0.6010 seconds
    STL Heap Sort: Duration: 0.5742 seconds
    Two sorting algorithms are correct.
    
    Test double:
    My Heap Sort: Duration: 0.6939 seconds
    STL Heap Sort: Duration: 0.6015 seconds
    Two sorting algorithms are correct.
    
    Testing with Order: DESC, Repeat: YES
    Test integer:
    My Heap Sort: Duration: 0.6129 seconds
    STL Heap Sort: Duration: 0.6417 seconds
    Two sorting algorithms are correct.
    
    Test double:
    My Heap Sort: Duration: 0.6897 seconds
    STL Heap Sort: Duration: 0.6178 seconds
    Two sorting algorithms are correct.
    
    Testing with Order: RANDOM, Repeat: NO
    Test integer:
    My Heap Sort: Duration: 2.0687 seconds
    STL Heap Sort: Duration: 2.0665 seconds
    Two sorting algorithms are correct.
    
    Test double:
    My Heap Sort: Duration: 2.0905 seconds
    STL Heap Sort: Duration: 1.9466 seconds
    Two sorting algorithms are correct.
    
    Testing with Order: RANDOM, Repeat: YES
    Test integer:
    My Heap Sort: Duration: 1.8029 seconds
    STL Heap Sort: Duration: 1.6197 seconds
    Two sorting algorithms are correct.
    
    Test double:
    My Heap Sort: Duration: 2.0983 seconds
    STL Heap Sort: Duration: 1.9387 seconds
    Two sorting algorithms are correct.
\end{verbatim}

下面是时间对比表格:

\begin{table}[H]
    \centering
    \begin{tabular}{|c|c|c|}
        \hline
        数据类型 & 我的堆排序时间 (秒) & STL 堆排序时间 (秒) \\
        \hline
        \multicolumn{3}{|c|}{Order: ASC, Repeat: NO} \\
        \hline
        整数 & 0.5548 & 0.4965 \\
        浮点数 & 0.6206 & 0.4936 \\
        \hline
        \multicolumn{3}{|c|}{Order: ASC, Repeat: YES} \\
        \hline
        整数 & 0.5784 & 0.4909 \\
        浮点数 & 0.6434 & 0.4913 \\
        \hline
        \multicolumn{3}{|c|}{Order: DESC, Repeat: NO} \\
        \hline
        整数 & 0.6010 & 0.5742 \\
        浮点数 & 0.6939 & 0.6015 \\
        \hline
        \multicolumn{3}{|c|}{Order: DESC, Repeat: YES} \\
        \hline
        整数 & 0.6129 & 0.6417 \\
        浮点数 & 0.6897 & 0.6178 \\
        \hline
        \multicolumn{3}{|c|}{Order: RANDOM, Repeat: NO} \\
        \hline
        整数 & 2.0687 & 2.0665 \\
        浮点数 & 2.0905 & 1.9466 \\
        \hline
        \multicolumn{3}{|c|}{Order: RANDOM, Repeat: YES} \\
        \hline
        整数 & 1.8029 & 1.6197 \\
        浮点数 & 2.0983 & 1.9387 \\
        \hline
    \end{tabular}
    \caption{不同排列组合下的排序时间}
    \label{tab:sorting_times}
\end{table}

可以看到,首先我们的排序算法是正确的。
在开启O2优化的情况下,我们的堆排序算法的效率和STL堆排序算法的效率相差不大,
标准库的排序稍微快一些。浮点数的时间比整数稍微长一点,原因可能是浮点数据类型的比较
和交换操作比整数类型的操作耗时更长。

关于我们的堆排序和STL堆排序的时间差异,可能的原因包括

\begin{itemize}
    \item \textbf{实现细节}:STL的\texttt{std::make\_heap}和\texttt{std::sort\_heap}是经过高度优化的通用算法,适用于各种类型的数据和比较函数。它们在实现上可能包含了一些我们自定义实现中没有的优化。
    \item \textbf{内存访问模式}:STL算法可能在内存访问模式上进行了优化,减少了缓存未命中(cache miss)的情况,从而提高了性能。
    \item \textbf{编译器优化}:编译器在优化STL库代码时,可能会进行一些特定的优化,而这些优化在我们自定义实现的代码中可能无法完全发挥作用。值得一提的是,在不开启优化的情况下,STL堆排序的性能会比我们自定义实现的堆排序更差。
    \item \textbf{算法复杂度}:虽然堆排序的时间复杂度在最坏情况下是$O(n \log n)$,但常数因子也会影响实际运行时间。STL实现可能在常数因子上进行了优化。
\end{itemize}

\end{document}



%%% Local Variables: 
%%% mode: latex
%%% TeX-master: t
%%% End:
