\documentclass[UTF8]{ctexart}
\usepackage{geometry, CJKutf8}
\geometry{margin=2cm, vmargin={0pt,1cm}} % 调整页面边距
\setlength{\topmargin}{-1.64723cm} % 调整 topmargin
\setlength{\headheight}{12.64723pt} % 设置 headheight
\setlength{\paperheight}{29.7cm}
\setlength{\textheight}{25.3cm}

% useful packages.
\usepackage{amsfonts}
\usepackage{listingsutf8}
\usepackage{amsmath}
\usepackage{amssymb}
\usepackage{amsthm}
\usepackage{enumerate}
\usepackage{graphicx}
\usepackage{multicol}
\usepackage{fancyhdr}
\usepackage{layout}
\usepackage{listings}
\usepackage{float, caption}
\usepackage{fontspec}
\setmonofont{Noto Sans Mono}

\lstset{
    basicstyle=\ttfamily, basewidth=0.5em, inputencoding=utf8, showstringspaces=false, showspaces=false
}

% some common command
\newcommand{\dif}{\mathrm{d}}
\newcommand{\avg}[1]{\left\langle #1 \right\rangle}
\newcommand{\difFrac}[2]{\frac{\dif #1}{\dif #2}}
\newcommand{\pdfFrac}[2]{\frac{\partial #1}{\partial #2}}
\newcommand{\OFL}{\mathrm{OFL}}
\newcommand{\UFL}{\mathrm{UFL}}
\newcommand{\fl}{\mathrm{fl}}
\newcommand{\op}{\odot}
\newcommand{\Eabs}{E_{\mathrm{abs}}}
\newcommand{\Erel}{E_{\mathrm{rel}}}

\begin{document}

\sloppy % 或者使用 \raggedright

\pagestyle{fancy}
\fancyhead{}
\lhead{黎学圣, 3230102179}
\chead{数据结构与算法项目作业}
\rhead{Oct.30th, 2024}

\section{项目结构}

\begin{itemize}

  \item \texttt{main.cpp}:主程序入口
  \item \texttt{expression\_evaluator.h}:表达式求值器的定义
  \item \texttt{expression\_evaluator.cpp}:表达式求值器的实现
  \item \texttt{Makefile}:运行项目
  \item \texttt{test.py}:表达式求值对照
  \item \texttt{compare\_results.sh}:对照测试脚本

\end{itemize}

\section{Makefile的使用}

\texttt{Makefile} 文件用于管理项目的构建和运行。以下是各个目标的说明:

\begin{itemize}
  \item \texttt{all}:默认目标,编译生成可执行文件 \texttt{main}。
  \item \texttt{report}:使用 \texttt{xelatex} 编译 \texttt{report.tex} 生成报告。
  \item \texttt{run\_cpp}:运行生成的 C++ 可执行文件,并将输出重定向到 \texttt{cpp\_output.txt}。
  \item \texttt{valgrind\_cpp}:使用 \texttt{valgrind} 检查 C++ 程序的内存泄漏。
  \item \texttt{run\_python}:运行 Python 对照组脚本 \texttt{test.py},并将输出重定向到 \texttt{python\_output.txt}。
  \item \texttt{test}:依次运行 \texttt{run\_cpp} 和 \texttt{run\_python},然后调用 \texttt{compare\_results.sh} 脚本对比结果。
  \item \texttt{clean}:清理生成的文件,包括可执行文件、输出文件和报告生成的中间文件。
\end{itemize}

使用方法示例:

\begin{verbatim}
# 编译生成可执行文件
make

# 生成报告
make report

# 运行 C++ 程序
make run_cpp

# 运行 Python 脚本
make run_python

# 对比 C++ 和 Python 的结果
make test

# 清理生成的文件
make clean
\end{verbatim}

\section{函数清单}

\texttt{ExpressionEvaluator} 类提供了一系列静态函数,用于表达式的预处理、解析和求值。以下是各个函数的说明:

\begin{itemize}
  \item \texttt{static bool hasInvalidCharacters(const std::string\& expression)}:检测表达式中是否包含非法字符。
  \item \texttt{static bool areParenthesesBalanced(const std::string\& expression)}:检测表达式中的括号是否匹配。
  \item \texttt{static bool hasExtraDecimalPoint(const std::string\& expression)}:检测表达式中是否包含多余的小数点。
  \item \texttt{static bool hasInvalidOperatorSequence(const std::string\& expression)}:检测表达式中是否包含连续的运算符。
  \item \texttt{static std::string removeSpaces(const std::string\& expression)}:删除表达式中的空格。
  \item \texttt{static std::string addParentheses(const std::string\& expression)}:为表达式添加括号。
  \item \texttt{static size\_t findRightOperandEnd(const std::string\& expression, size\_t start)}:寻找表达式中右操作数的结束位置。
  \item \texttt{static std::string preprocessExpression(const std::string\& expression)}:预处理表达式。
  \item \texttt{static std::vector<std::string> tokenizeExpression(const std::string\& expression)}:将表达式拆分为标记。
  \item \texttt{static std::string infixToPostfix(const std::string\& expression)}:将中缀表达式转换为后缀表达式。
  \item \texttt{static double evaluatePostfix(const std::string\& postfixExpression)}:计算后缀表达式的值。
\end{itemize}

此外,类中还包含一些私有的辅助函数:

\begin{itemize}
  \item \texttt{static int getPrecedence(char op)}:获取运算符的优先级。
  \item \texttt{static bool isOperator(char c)}:判断字符是否为运算符。
  \item \texttt{static double applyOperation(double a, double b, char op)}:应用运算符对两个操作数进行计算。
\end{itemize}

\section{重要方法及思路说明}

一些小方法会在代码中进行注释

我们在进行表达式求值之前,首先对表达式进行多重预处理,之后的转换后缀表达式就更方便了。

\subsection{非法字符检测}

我们先直接排除非法字符,这样可以减少后续的处理。

检查是否包含非法字符(仅保留数字、运算符、小数点和括号)

\texttt{if (!std::isspace(c) \&\& !std::isdigit(c) \&\& c != '+' \&\& c != '-' \&\& c != '*' \&\& c != '/' \&\& c != '(' \&\& c != ')' \&\& c != 'e' \&\& c != '.')}

\subsection{括号匹配检测}

使用栈的思想,遇到左括号入栈,遇到右括号出栈,最后栈为空则括号匹配

\subsection{多余小数点检测}

为了检测表达式中是否包含多余的小数点,我们定义了 \texttt{hasExtraDecimalPoint} 函数。该函数遍历表达式中的每个字符,并根据以下规则进行检测:

\begin{itemize}
  \item 如果字符是数字,则标记当前处于一个数字部分。
  \item 如果字符是小数点,则检查当前数字部分是否已经包含小数点。如果是,则返回 \texttt{true},表示存在多余的小数点。
  \item 如果遇到其他字符,则结束当前数字部分的检查,重置标记。
\end{itemize}

以下是 \texttt{hasExtraDecimalPoint} 函数的实现:

\begin{lstlisting}[language=C++, breaklines=true]
bool ExpressionEvaluator::hasExtraDecimalPoint(const std::string& expression) {
    bool inNumber = false;  // 是否在一个数字部分中
    bool hasDecimalPoint = false;  // 当前数字部分是否已有小数点

    for (char c : expression) {
        if (std::isdigit(c)) {
            inNumber = true;  // 开始一个新的数字部分
        } else if (c == '.') {
            if (hasDecimalPoint) {
                return true;  // 如果已经有小数点,则多余
            }
            hasDecimalPoint = true;  // 标记当前数字部分有小数点
        } else {
            // 遇到其他字符,结束当前数字部分的检查
            inNumber = false;
            hasDecimalPoint = false;
        }
    }

    return false;  // 未检测到多余小数点
}
\end{lstlisting}

该方法确保表达式中的每个数字部分最多只包含一个小数点,从而保证表达式的合法性。

\subsection{移出空格等空白字符}

是\texttt{removeSpaces}函数,直接遍历表达式,将空格等空白字符移出

\subsection{添加括号}

为了方便后续的处理,我们在表达式的两端添加括号。这样可以确保表达式的正确性,同时也方便我们进行后续的处理,
不需要考虑边界情况。

\subsection{非法连续运算符检测}

这是比较无脑的方法,但能够有效排除非法情况。

我们先列了一张连续运算符的表,然后遍历表达式,如果遇到连续运算符,就返回非法。

\begin{table}[H]
\centering
\begin{tabular}{c|c|c|c|c|c|c|c|c|c}
    & + & - & * & / & ( & ) & e & . & d \\
    \hline
    + & 1 & 1 & 0 & 0 & 1 & 0 & 1 & 1 & 1 \\
    - & 1 & 1 & 0 & 0 & 1 & 0 & 1 & 1 & 1 \\
    * & 1 & 1 & 0 & 0 & 1 & 0 & 1 & 1 & 1 \\
    / & 1 & 1 & 0 & 0 & 1 & 0 & 1 & 1 & 1 \\
    ( & 1 & 1 & 0 & 0 & 1 & 0 & 1 & 1 & 1 \\
    ) & 1 & 1 & 1 & 1 & 0 & 1 & 1 & 0 & 0 \\
    e & 0 & 1 & 0 & 0 & 1 & 0 & 0 & 1 & 1 \\
    . & 0 & 0 & 0 & 0 & 0 & 0 & 0 & 0 & 1 \\
    d & 1 & 1 & 1 & 1 & 0 & 1 & 1 & 1 & 1 \\
\end{tabular}
\caption{连续运算符检测表}
\end{table}

其中我们把 \texttt{e} 视为运算符,因为它是科学计数法的一部分。
表格中的 \texttt{d} 表示数字。
比如我们允许连续的加法和减法运算,后续会被预处理,但不允许 \texttt{+*} 或 \texttt{)()} 这种情况。

我们直接列举 
\begin{lstlisting}
    std::vector<std::string> invalidSequence = {
        "+*", "+/", "+)",
        "-*", "-/", "-)",
        "**", "*/", "*)",
        "/*", "//", "/)",
        "e*", "e/", "ee", "e)", "e+",
        "(*", "(/", "()",
        ")(", ").",
        ".+", ".-", ".*", "./", ".e",
        ".)", ".(", "..",
        "0(", "1(", "2(", "3(", "4(", "5(", "6(", "7(", "8(", "9(",
        ")0", ")1", ")2", ")3", ")4", ")5", ")6", ")7", ")8", ")9",
    };
\end{lstlisting}

\subsection{预处理表达式}

为最复杂但重要的一步,我们分四步进行预处理:

\subsubsection{第一步:移除空白字符,添加括号并检测合法性}

\begin{lstlisting}[language=c++, breaklines=true]
    std::string cleanedExpression = ExpressionEvaluator::addParentheses(expression);

    // 检查合法性
    if (!hasInvalidCharacters(cleanedExpression)) {
        throw std::invalid_argument("表达式包含非法字符");
    }
    if (!areParenthesesBalanced(cleanedExpression)) {
        throw std::invalid_argument("表达式中的括号不平衡");
    }
    if (hasExtraDecimalPoint(cleanedExpression)) {
        throw std::invalid_argument("表达式包含多余的小数点");
    }
    if (hasInvalidOperatorSequence(cleanedExpression)) {
        throw std::invalid_argument("表达式包含无效的运算符序列");
    }
\end{lstlisting}

\subsubsection{第二步:处理连续加减运算符}

目的是将连续的加减运算符转换为单个运算符,这样可以简化后续的处理,比如+-简化成-,--简化成+。

在表达式中,连续的加减运算符需要进行特殊处理。我们通过遍历表达式中的每个字符,检测连续的加减运算符,并根据其数量进行处理。以下是具体的实现步骤:

1. 初始化变量 \texttt{signCount} 用于记录连续减号的数量,\texttt{inConsecutiveSigns} 用于标记是否处于连续加减运算符的检测中。
2. 遍历表达式中的每个字符:
   \begin{itemize}
     \item 如果字符是加号或减号:
       \begin{itemize}
         \item 如果当前不在连续加减运算符的检测中,开始新的检测,并重置 \texttt{signCount}。
         \item 如果字符是减号,增加 \texttt{signCount}。
       \end{itemize}
     \item 如果字符不是加号或减号:
       \begin{itemize}
         \item 如果当前在连续加减运算符的检测中,根据 \texttt{signCount} 的奇偶性决定最终的运算符,并结束检测。
         \item 将当前字符添加到结果字符串中。
       \end{itemize}
   \end{itemize}
3. 处理表达式末尾的连续加减运算符。

以下是处理连续加减运算符的代码实现:

\begin{lstlisting}[language=C++, breaklines=true]
std::string result_1;
int signCount = 0;
bool inConsecutiveSigns = false;

for (size_t i = 0; i < cleanedExpression.size(); ++i) {
    char c = cleanedExpression[i];

    if (c == '+' || c == '-') {
        if (!inConsecutiveSigns) {
            inConsecutiveSigns = true;
            signCount = 0;
        }
        signCount += (c == '-') ? 1 : 0;
    } else {
        if (inConsecutiveSigns) {
            result_1 += (signCount % 2 == 0) ? '+' : '-';
            inConsecutiveSigns = false;
            signCount = 0;
        }
        result_1 += c;
    }
}

// 遇到结尾的连续符号
if (inConsecutiveSigns) {
    result_1 += (signCount % 2 == 0) ? '+' : '-';
}
\end{lstlisting}

\subsubsection{第三步:处理省略的运算符操作数}

在表达式中,有时会省略运算符的操作数,例如 \texttt{e} 后面没有数字,或者小数点前面没有数字。为了处理这些情况,我们需要对表达式进行进一步的预处理。以下是具体的实现步骤:

基于运算符合法性表格进行分步处理。

1. 初始化一个标记 \texttt{modified},用于记录当前扫描是否对表达式进行了修改。
2. 使用一个循环遍历表达式中的每个字符,并根据以下规则进行处理:
   \begin{itemize}
     \item 如果当前字符是运算符(如 \texttt{+}、\texttt{-}、\texttt{*}、\texttt{/}、\texttt{(}),检查下一个字符:
       \begin{itemize}
         \item 如果下一个字符是 \texttt{e},则在运算符后面添加 \texttt{1e}。
         \item 如果下一个字符是小数点,则在运算符后面添加 \texttt{0.}。
         \item 如果当前字符是 \texttt{*} 或 \texttt{/},且下一个字符是 \texttt{+},则跳过 \texttt{+}。
         \item 如果当前字符是 \texttt{(},且下一个字符是 \texttt{+} 或 \texttt{-},则在 \texttt{(} 后面添加 \texttt{0}。
       \end{itemize}
     \item 如果当前字符是 \texttt{e},且下一个字符是小数点,则在 \texttt{e} 后面添加 \texttt{0.}。
   \end{itemize}
3. 如果在一次扫描中对表达式进行了修改,则继续进行下一次扫描,直到表达式在一次扫描中没有任何变化为止。

代码实现可直接查看 \texttt{expression\_evaluator.cpp} 源代码。

这种方法类似于冒泡排序的优化,通过在每次扫描中记录是否进行了修改,如果没有修改则停止循环,从而提高了处理效率。

\subsubsection{第四步:处理负号}

这一步单独处理,是因为负号直接加 0 会导致优先级出错,比如 \texttt{*-2} 会被处理成 \texttt{*0-2},这是不对的。

为了正确处理负号,我们需要在负号前添加括号,并在括号内添加 0,从而保证运算的优先级。例如,\texttt{*-2} 应该被处理成 \texttt{*(0-2)}。类似地,\texttt{7*-(2.5+3)} 需要处理成 \texttt{7*(-(2.5+3))},\texttt{7*-3e-6*2} 需要处理成 \texttt{7*(0-3e(0-6))*2}。

具体的实现步骤如下:

1. 初始化一个变量 \texttt{previousResult} 用于保存当前结果,以便检查是否有变更。

2. 使用一个循环遍历表达式中的每个字符,并根据以下规则进行处理:
   \begin{itemize}
     \item 如果当前字符是 \texttt{*} 或 \texttt{/},且下一个字符是 \texttt{-},则在运算符后面添加 \texttt{(0-},并处理负号后的数字或括号。
     \item 如果当前字符是 \texttt{e},且下一个字符是 \texttt{-},则在 \texttt{e} 后面添加 \texttt{(0-},并处理负号后的数字或括号。因为e的优先级最高,处理方法比乘除简单些。
     \item 其他情况下,直接将当前字符添加到结果字符串中。
   \end{itemize}
3. 如果在一次扫描中对表达式进行了修改,则继续进行下一次扫描,直到表达式在一次扫描中没有任何变化为止。

为了处理负号后的数字或括号,我们定义了辅助方法 \texttt{findRightOperandEnd},用于寻找右操作数的结束位置。该方法的实现步骤如下:

1. 初始化一个变量 \texttt{i} 为起始位置。

2. 如果当前字符是运算符,且其后接 \texttt{(},则找到对应的闭括号位置。

3. 如果当前字符是运算符,且其后接另一个运算符,则递归调用 \texttt{findRightOperandEnd} 方法。

4. 遍历操作数,可能包含小数点,直到找到右操作数的结束位置。

以下是 \texttt{findRightOperandEnd} 方法的代码实现:

\begin{lstlisting}[language=C++, breaklines=true]
size_t ExpressionEvaluator::findRightOperandEnd(const std::string& expression, size_t start) {
    size_t i = start;

    // 检查是否为运算符,且其后接 "(",找到闭括号位置
    if (isOperator(expression[i])) {
        if (expression[i + 1] == '(') {
            int openParens = 1;
            i += 2;
            while (i < expression.size() && openParens > 0) {
                if (expression[i] == '(') openParens++;
                else if (expression[i] == ')') openParens--;
                i++;
            }
            return i;
        } else if (isOperator(expression[i + 1])) {
            return findRightOperandEnd(expression, i + 1);
        }
        i++;  // 跳过以检查后面的数
    }

    // 遍历操作数,可能包含小数点 
    while (i < expression.size() && (std::isdigit(expression[i]) || expression[i] == '.')) {
        i++;
    }

    return i;
}
\end{lstlisting}

通过上述方法,我们可以正确处理表达式中的负号,确保运算的优先级。

具体的代码实现请参见 \texttt{expression\_evaluator.cpp} 文件中的相关部分。

\subsection{将中缀表达所进行分组}

目的是简化转换的处理,将double数值,运算符,括号分开,放到一个vector中。

\subsection{中缀表达式转后缀表达式}

在表达式求值过程中,将中缀表达式转换为后缀表达式(逆波兰表达式)是一个重要步骤。后缀表达式的优点在于不需要括号来表示操作符的优先级,计算机可以更容易地进行求值。

以下是将中缀表达式转换为后缀表达式的具体步骤:

1. 初始化一个栈用于存储操作符和括号,以及一个字符串用于存储后缀表达式。
2. 遍历中缀表达式中的每个标记(token):
   \begin{itemize}
     \item 如果标记是数值,直接添加到后缀表达式中。
     \item 如果标记是左括号,直接入栈。
     \item 如果标记是右括号,弹出栈顶的操作符并添加到后缀表达式中,直到遇到左括号。
     \item 如果标记是操作符,弹出栈顶的操作符并添加到后缀表达式中,直到栈顶的操作符优先级低于当前操作符,然后将当前操作符入栈。
   \end{itemize}
3. 遍历结束后,将栈中剩余的所有操作符弹出并添加到后缀表达式中。

以下是 \texttt{infixToPostfix} 函数的代码实现:

\begin{lstlisting}[language=C++, breaklines=true]
std::string ExpressionEvaluator::infixToPostfix(const std::string& expression) {
    std::vector<std::string> tokens = tokenizeExpression(expression);
    std::stack<std::string> stack;
    std::string postfix;

    for (const std::string& token : tokens) {
        // 数值直接输出到后缀表达式
        if (std::isdigit(token[0]) || (token.size() > 1 && (std::isdigit(token[1]) || token[0] == '.'))) {
            postfix += token + " ";
        }
        // 左括号直接入栈
        else if (token == "(") {
            stack.push(token);
        }
        // 遇到右括号,弹出直到左括号
        else if (token == ")") {
            while (!stack.empty() && stack.top() != "(") {
                postfix += stack.top() + " ";
                stack.pop();
            }
            if (!stack.empty()) stack.pop(); // 弹出左括号
        }
        // 遇到运算符
        else if (isOperator(token[0])) {
            while (!stack.empty() && getPrecedence(stack.top()[0]) >= getPrecedence(token[0])) {
                postfix += stack.top() + " ";
                stack.pop();
            }
            stack.push(token);
        }
    }

    // 将栈中剩余的所有运算符弹出到后缀表达式
    while (!stack.empty()) {
        postfix += stack.top() + " ";
        stack.pop();
    }

    return postfix;
}
\end{lstlisting}

通过上述步骤,我们可以将中缀表达式转换为后缀表达式,从而简化表达式的求值过程。

\subsection{计算后缀表达式的值}

在将中缀表达式转换为后缀表达式后,我们需要计算后缀表达式的值。计算后缀表达式的过程如下:

1. 初始化一个栈用于存储操作数。
2. 遍历后缀表达式中的每个标记(token):
   \begin{itemize}
     \item 如果标记是数值,将其转换为双精度浮点数并压入栈中。
     \item 如果标记是操作符,从栈中弹出两个操作数,执行相应的操作,并将结果压入栈中。
   \end{itemize}
3. 遍历结束后,栈中应只剩下一个元素,即后缀表达式的计算结果。

以下是 \texttt{evaluatePostfix} 函数的代码实现:

\begin{lstlisting}[language=C++, breaklines=true]
double ExpressionEvaluator::evaluatePostfix(const std::string& postfixExpression) {
    std::istringstream tokens(postfixExpression);
    std::stack<double> stack;
    std::string token;

    while (tokens >> token) {
        // 判断是否为数字
        if (std::isdigit(token[0]) || (token.size() > 1 && token[0] == '-')) {
            stack.push(std::stod(token));
        } 
        // 判断是否为运算符
        else if (isOperator(token[0])) {
            if (stack.size() < 2) {
                throw std::invalid_argument("Invalid postfix expression");
            }
            double right = stack.top(); stack.pop();
            double left = stack.top(); stack.pop();

            // 使用 applyOperation 执行对应操作
            double result = applyOperation(left, right, token[0]);
            stack.push(result);
        } else {
            throw std::invalid_argument("Unknown token in postfix expression: " + token);
        }
    }

    if (stack.size() != 1) {
        throw std::invalid_argument("Invalid postfix expression");
    }

    return stack.top();
}
\end{lstlisting}

通过上述步骤,我们可以计算后缀表达式的值。具体的实现步骤如下:

1. 使用 \texttt{std::istringstream} 将后缀表达式分割为标记。
2. 遍历每个标记:
   \begin{itemize}
     \item 如果标记是数值,使用 \texttt{std::stod} 将其转换为双精度浮点数,并压入栈中。
     \item 如果标记是操作符,从栈中弹出两个操作数,使用 \texttt{applyOperation} 执行相应的操作,并将结果压入栈中。
     \item 如果标记既不是数值也不是操作符,抛出异常。
   \end{itemize}
3. 遍历结束后,检查栈中是否只剩下一个元素。如果是,则返回该元素作为计算结果;否则,抛出异常。

通过这种方法,我们可以有效地计算后缀表达式的值,从而完成表达式求值的过程。

\subsection{进行测试}

我们在 \texttt{test.py} 中编写了一系列测试用例,用于对照 C++ 程序的输出结果。

进行测试的表达式示例如下:

\begin{lstlisting}[language=Python, breaklines=true]
    test_expressions = [
        "3.5 + 4.7e2 * +-.123 + (2 - -8) * 3",                
        "-3.5e-2 + 4 * -+-(2.5 + 5.5)",                       
        "+3.5e2 + +4.7e-2 * --5.5 + (+.123 - -8) * -3",       
        "+3.5 + -4.7e2 * -5 + (2 - -8) * +3 / -2 - +1",       
        "-3.5e2 + +4.7e-2 * -+-0.123 + (2 - 8) * 3 / 2 - +1 + 6e2", 
        "3.5e2 + 4.7e-2 * -.123 + +(2 - -8) * +3 / -2 - +1 + +6e-2", 
        "+3.5 + -4.7e2 * -5.5 + +(2 - 8) * -3 / 2 + +8e-2",   
        "--3.5 + +4 * -+-(2.5e-2 + .123)",                    
        "++3.5 + -4.7e2 * -.5 + +2e2 / 4 - .123e-2",          
        "+3.5 * (4.7e2 - +3e2 / +(2e-2 + .5))"                
    ]
\end{lstlisting}

下面是 \texttt{compare\_results.sh} 脚本的部分代码:

\begin{lstlisting}[language=bash, breaklines=true]
    #!/bin/bash

    # Extract C++ results
    cpp_results=$(grep -oP '计算结果: \K[-+]?([0-9]*\.[0-9]+|[0-9]+)' cpp_output.txt)
    
    # Extract Python results
    python_results=$(grep -oP 'Result: \K[-+]?([0-9]*\.[0-9]+|[0-9]+)' python_output.txt)
    
    # Compare each result line by line
    cpp_array=($cpp_results)
    python_array=($python_results)
    
    echo "Comparing results (C++ - Python):"
    num_tests=${#cpp_array[@]}
    for ((i=0; i<$num_tests; i++)); do
        cpp_val=${cpp_array[i]}
        python_val=${python_array[i]}
    
        # Calculate the difference
        difference=$(echo "$cpp_val - $python_val" | bc -l)
    
        # Print comparison result
        echo "Test $((i+1)): C++ result = $cpp_val, Python result = $python_val, Difference = $difference"
    done
\end{lstlisting}    

测试结果如下:

\begin{lstlisting}[breaklines=true]
    g++ -std=c++17 -g -O2 -o main main.cpp
    ./main > cpp_output.txt
    python3 test.py > python_output.txt
    Comparing results (C++ - Python):
    Test 1: C++ result = -24.31, Python result = -24.310000000000002, Difference = .000000000000002
    Test 2: C++ result = 31.965, Python result = 31.965, Difference = 0
    Test 3: C++ result = 325.889, Python result = 325.8895, Difference = -.0005
    Test 4: C++ result = 2337.5, Python result = 2337.5, Difference = 0
    Test 5: C++ result = 240.006, Python result = 240.005781, Difference = .000219
    Test 6: C++ result = 334.054, Python result = 334.054219, Difference = -.000219
    Test 7: C++ result = 2597.58, Python result = 2597.58, Difference = 0
    Test 8: C++ result = 4.092, Python result = 4.092, Difference = 0
    Test 9: C++ result = 288.499, Python result = 288.49877, Difference = .00023
    Test 10: C++ result = -374.231, Python result = -374.23076923076917, Difference = -.00023076923083
\end{lstlisting}    

可以看到,C++ 程序和 Python 脚本的计算结果基本一致,差异非常小,符合预期。

我们顺便给出C++的输出情况:

\begin{lstlisting}[breaklines=true]
    原始表达式: 3.5 + 4.7e2 * +-.123 + (2 - -8) * 3
    预处理后: (3.5+4.7e2*(0-0.123)+(2+8)*3)
    后缀表达式: 3.5 4.7 2 e 0 0.123 - * + 2 8 + 3 * + 
    计算结果: -24.31
    
    原始表达式: -3.5e-2 + 4 * -+-(2.5 + 5.5)
    预处理后: (0-3.5e(0-2)+4*(2.5+5.5))
    后缀表达式: 0 3.5 0 2 - e - 4 2.5 5.5 + * + 
    计算结果: 31.965
    
    原始表达式: +3.5e2 + +4.7e-2 * --5.5 + (+.123 - -8) * -3
    预处理后: (0+3.5e2+4.7e(0-2)*5.5+(0+0.123+8)*(0-3))
    后缀表达式: 0 3.5 2 e + 4.7 0 2 - e 5.5 * + 0 0.123 + 8 + 0 3 - * + 
    计算结果: 325.889
    
    原始表达式: +3.5 + -4.7e2 * -5 + (2 - -8) * +3 / -2 - +1
    预处理后: (0+3.5-4.7e2*(0-5)+(2+8)*3/(0-2)-1)
    后缀表达式: 0 3.5 + 4.7 2 e 0 5 - * - 2 8 + 3 * 0 2 - / + 1 - 
    计算结果: 2337.5
    
    原始表达式: -3.5e2 + +4.7e-2 * -+-0.123 + (2 - 8) * 3 / 2 - +1 + 6e2
    预处理后: (0-3.5e2+4.7e(0-2)*0.123+(2-8)*3/2-1+6e2)
    后缀表达式: 0 3.5 2 e - 4.7 0 2 - e 0.123 * + 2 8 - 3 * 2 / + 1 - 6 2 e + 
    计算结果: 240.006
    
    原始表达式: 3.5e2 + 4.7e-2 * -.123 + +(2 - -8) * +3 / -2 - +1 + +6e-2
    预处理后: (3.5e2+4.7e(0-2)*(0-0.123)+(2+8)*3/(0-2)-1+6e(0-2))
    后缀表达式: 3.5 2 e 4.7 0 2 - e 0 0.123 - * + 2 8 + 3 * 0 2 - / + 1 - 6 0 2 - e + 
    计算结果: 334.054
    
    原始表达式: +3.5 + -4.7e2 * -5.5 + +(2 - 8) * -3 / 2 + +8e-2
    预处理后: (0+3.5-4.7e2*(0-5.5)+(2-8)*(0-3/2)+8e(0-2))
    后缀表达式: 0 3.5 + 4.7 2 e 0 5.5 - * - 2 8 - 0 3 2 / - * + 8 0 2 - e + 
    计算结果: 2597.58
    
    原始表达式: --3.5 + +4 * -+-(2.5e-2 + .123)
    预处理后: (0+3.5+4*(2.5e(0-2)+0.123))
    后缀表达式: 0 3.5 + 4 2.5 0 2 - e 0.123 + * + 
    计算结果: 4.092
    
    原始表达式: ++3.5 + -4.7e2 * -.5 + +2e2 / 4 - .123e-2
    预处理后: (0+3.5-4.7e2*(0-0.5)+2e2/4-0.123e(0-2))
    后缀表达式: 0 3.5 + 4.7 2 e 0 0.5 - * - 2 2 e 4 / + 0.123 0 2 - e - 
    计算结果: 288.499
    
    原始表达式: +3.5 * (4.7e2 - +3e2 / +(2e-2 + .5))
    预处理后: (0+3.5*(4.7e2-3e2/(2e(0-2)+0.5)))
    后缀表达式: 0 3.5 4.7 2 e 3 2 e 2 0 2 - e 0.5 + / - * + 
    计算结果: -374.231
\end{lstlisting}    

\end{document}