\documentclass[UTF8]{ctexart}
\usepackage{geometry, CJKutf8}
\geometry{margin=1.5cm, vmargin={0pt,1cm}}
\setlength{\topmargin}{-1.64723cm} % 调整 topmargin
\setlength{\headheight}{12.64723pt} % 设置 headheight
\setlength{\paperheight}{29.7cm}
\setlength{\textheight}{25.3cm}

% useful packages.
\usepackage{amsfonts}
\usepackage{amsmath}
\usepackage{amssymb}
\usepackage{amsthm}
\usepackage{enumerate}
\usepackage{graphicx}
\usepackage{multicol}
\usepackage{fancyhdr}
\usepackage{layout}
\usepackage{listings}
\usepackage{float, caption}

\lstset{
    basicstyle=\ttfamily, basewidth=0.5em
}

% some common command
\newcommand{\dif}{\mathrm{d}}
\newcommand{\avg}[1]{\left\langle #1 \right\rangle}
\newcommand{\difFrac}[2]{\frac{\dif #1}{\dif #2}}
\newcommand{\pdfFrac}[2]{\frac{\partial #1}{\partial #2}}
\newcommand{\OFL}{\mathrm{OFL}}
\newcommand{\UFL}{\mathrm{UFL}}
\newcommand{\fl}{\mathrm{fl}}
\newcommand{\op}{\odot}
\newcommand{\Eabs}{E_{\mathrm{abs}}}
\newcommand{\Erel}{E_{\mathrm{rel}}}

\begin{document}

\pagestyle{fancy}
\fancyhead{}
\lhead{黎学圣, 3230102179}
\chead{数据结构与算法第四次作业}
\rhead{Oct.16th, 2024}

\section{链表实现的函数清单}

以下是链表实现的函数清单:

\subsection{构造函数和析构函数}
\begin{itemize}
  \item \texttt{List()}
  \item \texttt{List(std::initializer\_list<T> il)}
  \item \texttt{List(const List \&rhs)}
  \item \texttt{List(List \&\&rhs)}
  \item \texttt{\textasciitilde List()}
\end{itemize}

\subsection{赋值运算符}
\begin{itemize}
  \item \texttt{List \&operator=(List copy)}
\end{itemize}

\subsection{迭代器相关函数}
\begin{itemize}
  \item \texttt{iterator begin()}
  \item \texttt{const\_iterator begin() const}
  \item \texttt{iterator end()}
  \item \texttt{const\_iterator end() const}
\end{itemize}

\subsection{容量相关函数}
\begin{itemize}
  \item \texttt{int size() const}
  \item \texttt{bool empty() const}
  \item \texttt{void clear()}
\end{itemize}

\subsection{元素访问函数}
\begin{itemize}
  \item \texttt{T \&front()}
  \item \texttt{const T \&front() const}
  \item \texttt{T \&back()}
  \item \texttt{const T \&back() const}
\end{itemize}

\subsection{修改器函数}
\begin{itemize}
  \item \texttt{void push\_front(const T \&elem)}
  \item \texttt{void push\_front(T \&\&elem)}
  \item \texttt{void push\_back(const T \&elem)}
  \item \texttt{void push\_back(T \&\&elem)}
  \item \texttt{void pop\_front()}
  \item \texttt{void pop\_back()}
  \item \texttt{iterator insert(iterator itr, const T \&elem)}
  \item \texttt{iterator insert(iterator itr, T \&\&elem)}
  \item \texttt{iterator erase(iterator itr)}
  \item \texttt{iterator erase(iterator from, iterator to)}
\end{itemize}

\subsection{私有成员函数}
\begin{itemize}
  \item \texttt{void init()}
\end{itemize}

\section{重要函数说明}

在本节中,我们将详细说明链表实现中的两个重要函数:\texttt{insert} 和 \texttt{erase}。

\subsection{insert 函数}

\texttt{insert} 函数用于在链表的指定位置插入一个元素。它有两个重载版本,一个接受左值引用,另一个接受右值引用。

\begin{itemize}
  \item \texttt{iterator insert(iterator itr, const T \&elem)}: 该版本接受一个左值引用的元素,并在指定位置插入该元素。
  \item \texttt{iterator insert(iterator itr, T \&\&elem)}: 该版本接受一个右值引用的元素,并在指定位置插入该元素。
\end{itemize}

\textbf{实现原理}:

\begin{enumerate}
  \item 获取迭代器 \texttt{itr} 所指向的节点 \texttt{p}。
  \item 增加链表的大小 \texttt{theSize}。
  \item 创建一个新的节点,并将其插入到 \texttt{p} 的前面。
  \item 更新新节点的前驱和后继指针。
  \item 返回指向新插入节点的迭代器。
\end{enumerate}

\subsection{erase 函数}

\texttt{erase} 函数用于删除链表中指定位置的元素。它也有两个重载版本,一个删除单个元素,另一个删除一个范围内的元素。

\begin{itemize}
  \item \texttt{iterator erase(iterator itr)}: 该版本删除 \texttt{itr} 所指向的单个元素。
  \item \texttt{iterator erase(iterator from, iterator to)}: 该版本删除从 \texttt{from} 到 \texttt{to} 范围内的所有元素。
\end{itemize}

\textbf{实现原理}:

\begin{enumerate}
  \item 获取迭代器 \texttt{itr} 所指向的节点 \texttt{p}。
  \item 保存 \texttt{p} 的后继节点的迭代器作为返回值。
  \item 更新 \texttt{p} 的前驱和后继节点的指针,使它们跳过 \texttt{p}。
  \item 删除节点 \texttt{p} 并减少链表的大小 \texttt{theSize}。
  \item 返回指向 \texttt{p} 后继节点的迭代器。
\end{enumerate}

\section{测试的设计与结果}

为了验证链表实现的正确性,我们编写了一系列测试用例,涵盖了链表的主要功能。以下是测试的具体步骤和结果说明:

\subsection{测试用例}

\begin{itemize}
  \item \textbf{push\_back 测试}:向链表末尾添加元素,并检查链表内容是否正确。
  \item \textbf{push\_front 测试}:向链表头部添加元素,并检查链表内容是否正确。
  \item \textbf{pop\_back 测试}:从链表末尾删除元素,并检查链表内容是否正确。
  \item \textbf{pop\_front 测试}:从链表头部删除元素,并检查链表内容是否正确。
  \item \textbf{size 测试}:检查链表的大小是否正确。
  \item \textbf{empty 测试}:检查链表是否为空。
  \item \textbf{insert 和 erase 测试}:在链表中插入和删除元素,并检查链表内容是否正确。
  \item \textbf{initializer\_list 构造函数测试}:使用初始化列表创建链表,并检查链表内容是否正确。
  \item \textbf{拷贝构造函数测试}:使用拷贝构造函数创建链表副本,并检查副本内容是否正确。
  \item \textbf{移动构造函数测试}:使用移动构造函数创建链表,并检查原链表是否为空。
  \item \textbf{赋值运算符测试}:使用赋值运算符复制链表,并检查复制后的链表内容是否正确。
  \item \textbf{clear 测试}:清空链表,并检查链表是否为空。
\end{itemize}

\subsection{测试结果}

所有测试用例均通过,具体结果如下:

\begin{itemize}
  \item \textbf{push\_back 测试}:链表内容为 \texttt{1 2 3}。
  \item \textbf{push\_front 测试}:链表内容为 \texttt{0 1 2 3}。
  \item \textbf{pop\_back 测试}:链表内容为 \texttt{0 1 2}。
  \item \textbf{pop\_front 测试}:链表内容为 \texttt{1 2}。
  \item \textbf{size 测试}:链表大小为 \texttt{2}。
  \item \textbf{empty 测试}:链表为空状态为 \texttt{false}。
  \item \textbf{insert 和 erase 测试}:链表内容为 \texttt{1 99 2},删除最后一个元素后为 \texttt{1 99}。
  \item \textbf{initializer\_list 构造函数测试}:链表内容为 \texttt{10 20 30 40}。
  \item \textbf{拷贝构造函数测试}:链表副本内容为 \texttt{10 20 30 40}。
  \item \textbf{移动构造函数测试}:链表内容为 \texttt{10 20 30 40},原链表为空。
  \item \textbf{赋值运算符测试}:链表内容为 \texttt{10 20 30 40},修改后为 \texttt{10 20 30 40 50},原链表内容不变。
  \item \textbf{clear 测试}:链表大小为 \texttt{0},链表为空状态为 \texttt{true}。
\end{itemize}

\subsection{Valgrind 测试结果}

测试结果以 valgrind 的结果呈现,说明我们也没有发生内存泄漏:

\begin{verbatim}
==176073== Memcheck, a memory error detector
==176073== Copyright (C) 2002-2024, and GNU GPL'd, by Julian Seward et al.
==176073== Using Valgrind-3.24.0.GIT and LibVEX; rerun with -h for copyright info
==176073== Command: ./List
==176073== 
Testing push_back...
1 2 3 
Testing push_front...
0 1 2 3 
Testing pop_back...
0 1 2 
Testing pop_front...
1 2 
List size: 2
Testing empty function...
List is empty: false
Testing insert and erase...
1 99 2 
1 99 
Testing initializer_list constructor...
10 20 30 40 
Testing copy constructor...
10 20 30 40 
Testing move constructor...
10 20 30 40 
After move, list3 size: 0
Testing assignment operator...
10 20 30 40 
After modifying list5...
list5: 10 20 30 40 50 
list4: 10 20 30 40 
Testing clear function...
After clearing, list5 size: 0
list5 is empty: true
==176073== 
==176073== HEAP SUMMARY:
==176073==     in use at exit: 0 bytes in 0 blocks
==176073==   total heap usage: 30 allocs, 30 frees, 75,424 bytes allocated
==176073== 
==176073== All heap blocks were freed -- no leaks are possible
==176073== 
==176073== For lists of detected and suppressed errors, rerun with: -s
==176073== ERROR SUMMARY: 0 errors from 0 contexts (suppressed: 0 from 0)
\end{verbatim}

通过这些测试,我们验证了链表实现的正确性和稳定性。

\end{document}

%%% Local Variables: 
%%% mode: latex
%%% TeX-master: t
%%% End:
