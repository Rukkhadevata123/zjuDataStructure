\documentclass[UTF8]{ctexart}
\usepackage{geometry, CJKutf8}
\geometry{margin=1.5cm, vmargin={0pt,1cm}}
\setlength{\topmargin}{-1.64723cm} % 调整 topmargin
\setlength{\headheight}{12.64723pt} % 设置 headheight
\setlength{\paperheight}{29.7cm}
\setlength{\textheight}{25.3cm}

% useful packages.
\usepackage{amsfonts}
\usepackage{listingsutf8}
\usepackage{amsmath}
\usepackage{amssymb}
\usepackage{amsthm}
\usepackage{enumerate}
\usepackage{graphicx}
\usepackage{multicol}
\usepackage{fancyhdr}
\usepackage{layout}
\usepackage{listings}
\usepackage{float, caption}
\usepackage{fontspec}
\setmonofont{Noto Sans Mono}

\lstset{
    basicstyle=\ttfamily, basewidth=0.5em, inputencoding=utf8, showstringspaces=false, showspaces=false
}

% some common command
\newcommand{\dif}{\mathrm{d}}
\newcommand{\avg}[1]{\left\langle #1 \right\rangle}
\newcommand{\difFrac}[2]{\frac{\dif #1}{\dif #2}}
\newcommand{\pdfFrac}[2]{\frac{\partial #1}{\partial #2}}
\newcommand{\OFL}{\mathrm{OFL}}
\newcommand{\UFL}{\mathrm{UFL}}
\newcommand{\fl}{\mathrm{fl}}
\newcommand{\op}{\odot}
\newcommand{\Eabs}{E_{\mathrm{abs}}}
\newcommand{\Erel}{E_{\mathrm{rel}}}

\begin{document}

\pagestyle{fancy}
\fancyhead{}
\lhead{黎学圣, 3230102179}
\chead{数据结构与算法第六次作业}
\rhead{\today}

\section{添加了balance方法的remove函数}

在本节中,我们介绍了添加了平衡方法的 `remove` 函数。该函数的实现基于 `another\_remove` 方法,并在删除节点后对树进行平衡操作,以确保树保持 AVL 树的性质。

实际上,这里的删除逻辑与之前的 `remove` 函数相同,只是在删除节点后,我们需要对树进行平衡操作。这里我们介绍了 `balance` 方法,用于对树进行平衡操作。

\subsection{remove函数实现原理}

`remove` 函数的实现原理如下:

1. 首先,调用 `another\_remove` 方法删除指定的节点。
2. 然后,调用 `balance` 方法对树进行平衡操作。

\subsection{another\_remove函数}

`another\_remove` 函数用于从树中删除指定的节点。其实现步骤如下:

\begin{enumerate}
    \item \textbf{节点不存在}:如果当前节点 \verb|t| 为 \verb|nullptr|,直接返回。
    \item \textbf{待删除值小于当前节点值}:在左子树中继续递归查找。
    \item \textbf{待删除值大于当前节点值}:在右子树中继续递归查找。
    \item \textbf{找到待删除节点}:
    \begin{enumerate}
        \item \textbf{有两个子节点}:
        \begin{itemize}
            \item 调用 \verb|removeMin| 方法从右子树中找到最小节点。
            \item 将当前节点 \verb|t| 替换为该最小节点。
            \item 重新连接左右子树。
        \end{itemize}
        \item \textbf{有一个子节点或无子节点}:
        \begin{itemize}
            \item 将 \verb|t| 指向其非空子节点或 \verb|nullptr|。
        \end{itemize}
    \end{enumerate}
\end{enumerate}

\subsection{balance函数}

`balance` 函数用于对树进行平衡操作。其实现步骤如下:

1. 如果当前节点为空,则直接返回。
2. 如果左子树的高度比右子树高出1以上,则进行左旋转或双旋转操作:
   \begin{itemize}
       \item 如果左子树的左子树高度大于等于左子树的右子树高度,则进行单次左旋转。
       \item 否则,进行双旋转(先右旋转左子树,再左旋转当前节点)。
   \end{itemize}
3. 如果右子树的高度比左子树高出1以上,则进行右旋转或双旋转操作:
   \begin{itemize}
       \item 如果右子树的右子树高度大于等于右子树的左子树高度,则进行单次右旋转。
       \item 否则,进行双旋转(先左旋转右子树,再右旋转当前节点)。
   \end{itemize}
4. 更新当前节点的高度。

通过上述步骤,`remove` 函数在删除节点后能够保持树的平衡,确保树的高度保持在对数级别,从而提高查找、插入和删除操作的效率。

\end{document}

%%% Local Variables: 
%%% mode: latex
%%% TeX-master: t
%%% End:
